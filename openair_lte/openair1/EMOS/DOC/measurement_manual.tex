\documentclass[10pt,a4paper]{article}
%\usepackage{epsfig,amsmath,amsfonts,url,subfigure,verbatim}

\begin{document}
\author{Florian Kaltenberger (\texttt{florian.kaltenberger@eurecom.fr})}
\title{EMOS Multiuser Measurement Manual}

\maketitle

\section{EMOS Multiuser Measurement Manual}

\subsection{Checklist}

Each user should have the following equipment

\begin{itemize}
\item  Laptop
\item  Power supply
\item Car power adaptor
\item GPS
\item CBMIMO1 Card (Availiable cards: 2, 6, 7, 11. Card 10 has a bad frequency response and card 12 crashes after some time)
\item 2 Batteries
\item Antennas
\item This Manual
\item Route and Time
\item Walkie Talkie
\end{itemize} 

\subsection{Startup}
\begin{itemize}
\item Login using the name written on the laptop (password {\tt linux})
\item Power up the CBMIMO1 card, connect the antennas to the two outer connectors on the card, and insert it in the card slot
\item Connect the USB GPS receiver
\item Open a terminal window and change directory:\\
      {\tt cd EMOS/emos\_gui}
\item Load the kernel module as root:\\
      {\tt sudo make install\_cbmimo1\_softmodem\_emos}\\
      (provide password {\tt linux} if asked)\\
      The leds on the card should now be ``moving''
\item Start the GPS daemon as root:\\
      {\tt sudo gpsd -f /dev/ttyUSB0} (provide password {\tt linux} if asked)
\item Start the GUI:\\
      {\tt cd src}\\
      {\tt ./emos\_gui -n4}
%Start with {\tt ./emos\_gui -nx}, where {\tt x} is the number of base station antennas.
\end{itemize}

\subsection{Operation}
\begin{itemize}

\item Turn on the GUI by clicking the ``PWR'' button. 

\item Make sure, there are no error messages.

\item If the terminal can synchronize to the base station, the displays should turn on and the two rightmost leds on the card should stay on. 

\item Bring the GUI into multiuser mode by pressing the ``terminal mode'' button. 

\item Make sure you set the terminal number to the number indicated on your laptop.

\item Now you are ready to turn on the recorder by pressing ``record''. The terminal now waits for the base station to signal the recording flag. Upon receiving the flag, the terminal will record the estimated channel. When the recording flag goes off, the data is stored to file, the file index is increased, and the terminal goes into standby again.
\end{itemize}

\subsection{Troubleshooting}

\paragraph{Card does not synchronize} 
Most probably you are in an area with no or too little signal. 
\begin{itemize}
\item Make sure you stay on the measurement routes. 
\item Make sure the antennas are connected correctly. 
\item Check for other error messages on the GUI.
\end{itemize}

\paragraph{Channel displays do not refresh} 
Most probably the card has crashed or there is a problem with the fifo. 
\begin{itemize}
\item Stop recording and exit the GUI
\item Remove the kernel module:\\
      {\tt cd ~/EMOS/emos\_gui}\\
      {\tt sudo make remove\_cbmimo1\_emos}
\item Remove the card and the power
\item Load the kernel module as described above
\item Restart the GUI
\end{itemize}

\paragraph{No GPS data availiable}
\begin{itemize}
\item Make sure the GPS is connected correctly and that it can see the satelites (led on GPS constantly on)
\item Have you started the GPS daemon {\tt gpsd -f /dev/ttyUSB0}
\item Restart the GUI
\end{itemize}

\paragraph{GUI does not record or records permanently}
\begin{itemize}
\item Make sure you are in multiuser mode and you turned on record
\item Make sure there is enough disk space
\item Make sure that the record path (``config'' button) is set to ~/EMOS/data/
\end{itemize}

\paragraph{Computer freezes} 
\begin{itemize}
\item Reboot
\end{itemize}

\end{document}