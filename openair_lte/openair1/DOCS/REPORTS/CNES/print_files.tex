\documentclass[a4paper,10pt]{article}

\usepackage[pdftex,dvips]{graphicx}
\usepackage{amsmath}
\usepackage[T1]{fontenc}
\usepackage{url}
\usepackage[top=2cm, bottom=2cm, left=2cm, right=2cm]{geometry}



\newcommand{\printfile}[2][]{
 \begin{minipage}{8cm}
  \centering
  %\includegraphics[width=8cm]{/extras/kaltenbe/CNES/emos_postprocessed_data/#2}
  %\includegraphics[width=8cm]{/emos/EMOS/#1}
  \url{#2}: #1

 \end{minipage}
}

%\addtolength{\textwidth}{3cm}
%\setlength{\marginparwidth}{0cm}
%\setlength{\hoffset}{0cm}

\begin{document}
% \section{Mode 1}
% 
% \printfile{./Mode1/20100526_mode1_parcours1_part4_part5/RX_RSSI_dBm_gps.jpg}
% \printfile{./Mode1/20100527_mode1_parcours1/RX_RSSI_dBm_gps.jpg}
% 
% \section{Mode 2}
% \printfile{./Mode2/20100510_mode2_parcours1_part1/RX_RSSI_dBm_gps.jpg}
% \printfile{./Mode2/20100510_mode2_parcours1_part2/RX_RSSI_dBm_gps.jpg}
% 
% \printfile{./Mode2/20100510_mode2_parcours1_part3.1/RX_RSSI_dBm_gps.jpg}
% \printfile{./Mode2/20100510_mode2_parcours1_part3.2/RX_RSSI_dBm_gps.jpg}
% 
% \printfile{./Mode2/20100511_mode2_parcours1_part4_5_6/RX_RSSI_dBm_gps.jpg}
% \printfile{./Mode2/20100511_mode2_parcours2_part5/RX_RSSI_dBm_gps.jpg}
% 
% \printfile{./Mode2/20100512_mode2_parcours2_part7/RX_RSSI_dBm_gps.jpg}
% %\printfile{./Mode2/20100512_mode2_parcours2_part7/nomadic/RX_RSSI_dBm_gps.jpg}
% \printfile{./Mode2/20100518_mode2_parcours1_part6.1/RX_RSSI_dBm_gps.jpg}
% %\printfile{./Mode2/20100518_mode2_parcours1_part6.1/nomadic/RX_RSSI_dBm_gps.jpg}
% 
% \printfile{./Mode2/20100518_mode2_parcours1_part6.2/RX_RSSI_dBm_gps.jpg}
% %\printfile{./Mode2/20100518_mode2_parcours1_part6.2/nomadic/RX_RSSI_dBm_gps.jpg}
% \printfile{./Mode2/20100518_mode2_parcours1_part6.3/RX_RSSI_dBm_gps.jpg}
% %\printfile{./Mode2/20100518_mode2_parcours1_part6.3/nomadic/RX_RSSI_dBm_gps.jpg}
% 
% \printfile{./Mode2/20100519_mode2_parcours1_part7.1/RX_RSSI_dBm_gps.jpg}
% \printfile{./Mode2/20100519_mode2_parcours1_part7.2/RX_RSSI_dBm_gps.jpg}
% %\printfile{./Mode2/20100519_mode2_parcours1_part7.2/nomadic/RX_RSSI_dBm_gps.jpg}
% 
% \printfile{./Mode2/20100519_mode2_parcours1_part7.3/RX_RSSI_dBm_gps.jpg}
% %\printfile{./Mode2/20100519_mode2_parcours1_part7.3/nomadic/RX_RSSI_dBm_gps.jpg}
% \printfile{./Mode2/20100520_mode2_parcours1_part4_part5/RX_RSSI_dBm_gps.jpg}
% 
% \printfile{./Mode2/20100520_mode2_parcours1_part9_parcours2_part1/RX_RSSI_dBm_gps.jpg}
% \printfile{./Mode2/20100521_mode2_parcours2_part1_part2/RX_RSSI_dBm_gps.jpg}
% 
% \printfile{./Mode2/20100524_mode2_parcours2_part3_part4.1/RX_RSSI_dBm_gps.jpg}
% \printfile{./Mode2/20100524_mode2_parcours2_part3_part4.2/RX_RSSI_dBm_gps.jpg}
% 
% \printfile{./Mode2/20100524_mode2_parcours2_part3_part4.3/RX_RSSI_dBm_gps.jpg}
% \printfile{./Mode2/20100525_mode2_parcours2_part4_part8_part9_part10.1/RX_RSSI_dBm_gps.jpg}
% 
% \printfile{./Mode2/20100525_mode2_parcours2_part4_part8_part9_part10.2/RX_RSSI_dBm_gps.jpg}
% \printfile{./Mode2/20100525_mode2_parcours2_part4_part8_part9_part10.3/RX_RSSI_dBm_gps.jpg}
% 
% \printfile{./Mode2/20100607_VTP_MODE2_PARCOURS1_S2_3/RX_RSSI_dBm_gps.jpg}
% \printfile{./Mode2/20100608_VTP_MODE2_PARCOURS1_S7_9/RX_RSSI_dBm_gps.jpg}
% 
% \printfile{./Mode2/20100608_VTP_MODE2_PARCOURS1_S8/RX_RSSI_dBm_gps.jpg}
% 
% \section{Mode 6}
% 
% \printfile{./Mode6/20100528_mode6_parcours1_section1/RX_RSSI_dBm_gps.jpg}
% \printfile{./Mode6/20100528_mode6_parcours1_section2/RX_RSSI_dBm_gps.jpg}
% 
% \printfile{./Mode6/20100608_VTP_MODE6_ZONES_PUSCH_PART1/RX_RSSI_dBm_gps.jpg}
% \printfile{./Mode6/20100608_VTP_MODE6_ZONES_PUSCH_PART2/RX_RSSI_dBm_gps.jpg}
% 
% \printfile{./Mode6/20100608_VTP_MODE6_ZONES_PUSCH_PART3/RX_RSSI_dBm_gps.jpg}
% \printfile{./Mode6/20100609_VTP_MODE6_ZONES_PUSCH_PART4/RX_RSSI_dBm_gps.jpg}
% 
% \printfile{./Mode6/20100610_VTP_MODE6_ZONES_PUSCH_UPDATE.1/RX_RSSI_dBm_gps.jpg}
% \printfile{./Mode6/20100610_VTP_MODE6_ZONES_PUSCH_UPDATE.2/RX_RSSI_dBm_gps.jpg}
% 
% \section{UL}
% \printfile{./Mode6/20100608_VTP_MODE6_ZONES_PUSCH_PART3/UL_RSSI_dBm_gps.jpg}
% \printfile{./Mode6/20100608_VTP_MODE6_ZONES_PUSCH_PART2/UL_RSSI_dBm_gps.jpg}
% 
% \printfile{./Mode6/20100608_VTP_MODE6_ZONES_PUSCH_PART1/UL_RSSI_dBm_gps.jpg}
% \printfile{./Mode6/20100609_VTP_MODE6_ZONES_PUSCH_PART4/UL_RSSI_dBm_gps.jpg}
% 
% \printfile{./Mode6/20100610_VTP_MODE6_ZONES_PUSCH_UPDATE.1/UL_RSSI_dBm_gps.jpg}
% \printfile{./Mode6/20100610_VTP_MODE6_ZONES_PUSCH_UPDATE.2/UL_RSSI_dBm_gps.jpg}
% 
% \section{UL ideal}
% \printfile{./Mode6/20100608_VTP_MODE6_ZONES_PUSCH_PART3/UL_throughput_ideal_2Rx_gps.jpg}
% \printfile{./Mode6/20100608_VTP_MODE6_ZONES_PUSCH_PART2/UL_throughput_ideal_2Rx_gps.jpg}
% 
% \printfile{./Mode6/20100608_VTP_MODE6_ZONES_PUSCH_PART1/UL_throughput_ideal_2Rx_gps.jpg}
% \printfile{./Mode6/20100609_VTP_MODE6_ZONES_PUSCH_PART4/UL_throughput_ideal_2Rx_gps.jpg}
% 
% \printfile{./Mode6/20100610_VTP_MODE6_ZONES_PUSCH_UPDATE.1/UL_throughput_ideal_2Rx_gps.jpg}
% \printfile{./Mode6/20100610_VTP_MODE6_ZONES_PUSCH_UPDATE.2/UL_throughput_ideal_2Rx_gps.jpg}

%\section{all plots}

\section{Introduction}

This document summarizes the results of the LTE measurement campaign that was conducted by Eurecom and Sogeti April-July 2010 for the CNES.

The Eurecom testbench implements the LTE 3GPP release 8.6 \cite{3gpp_docs} with 5 MHz bandwidth and TDD uplink-downlink configuration 3 (i.e., there are 6 downlink subframes and 3 uplink subframes in a frame of 10ms). Extended cyclic prefix is used in both UL and DL. The carrier frequency is 859.6 Mhz. 
 
The measurement methodology and specification of the post processing are described in \cite{measurments_spec}. The measurment sites are 
\begin{itemize}
 \item Bournazel (close to Cordes-sur-ciel)
 \item Penne
 \item Ambialet
\end{itemize}
The are described in more detail in \cite{cordes_desc,penne_desc,ambialet_desc}. A list of all the measurements taken is given in \cite{measurments_spreadsheet}. 

According to the test plan \cite{}, the recorded measurements are divided into several directories
\begin{center}
% use packages: array
\begin{tabular}{l|p{6cm}}
Type of measurement & Routes taken \\ 
\hline
Interference & exterior road \\ 
Coverage & exterior road \\ 
Mode1 & parcours 2 \\ 
Mode2 & parcours 1 and 2 \\ 
Mode6 & parcours 2, zones PUSCH \\ 
Mode2 update\footnote{only at the site in Bournazel} & After a bug\footnote{The receiver for the PBCH was always assuming transmission mode 1. This caused a slight degradation in the performance in mode2, but only at the cell edge (at high SNR, both receiver structures work fine).} in the reception of the PBCH in transmission mode 2 was discovered, some of the routes of mode2 were done again. This data is only used for the evaluation of the PBCH performance. 
\end{tabular}
\end{center}



\section{Interference measurements}

After setting up the eNB and UE at a specific cell site, interference measurement (IM) is the very first and most important step of measurement campaign. Interference measurements are conducted to make sure that under test frequency band is exclusively used for the campaign and no other operator is using this frequency band in a specific cell. The following two figures show the measured interference in dBm at the UE and at the eNB.


\printfile[Interference at eNb]{Interference/20100421 interference eNb + DL test/RX_I0_dBm.pdf}
\printfile[Interference at UE]{Interference/results/RX_RSSI_dBm_gps.jpg}


\section{RX RSSI}


The received signal strength indicator (RSSI) is a measure of the strength of signal which is received at the antenna. Rx RSSI of both UE and eNB measured during coverage runs and vehicular test procedures is plotted on the regional map of the specific cell as well as a function of time. We show the received signal strength in dBm for the coverage run, the measurements in mode 1, 2, and 6 as well as the uplink RSSI. The path loss evaluation includes the data of all the measurements.


\subsection{Coverage}
\printfile{Coverage/results/RX_RSSI_dBm_gps.jpg}
\printfile{Coverage/results/RX_RSSI_dBm.pdf}

\printfile{Coverage/results/UL_RSSI_dBm_gps.jpg}
\printfile{Coverage/results/UL_RSSI_dBm.pdf}

\subsection{Mode 1}
\printfile{Mode1/results/RX_RSSI_dBm_gps.jpg}
\printfile{Mode1/results/RX_RSSI_dBm.pdf}

\printfile{Mode1/results/UL_RSSI_dBm_gps.jpg}
\printfile{Mode1/results/UL_RSSI_dBm.pdf}

\subsection{Mode2}





\printfile{Mode2/results/RX_RSSI_dBm_gps.jpg}
\printfile{Mode2/results/RX_RSSI_dBm.pdf}

\printfile{Mode2_update/results/UL_RSSI_dBm_gps.jpg}
\printfile{Mode2_update/results/UL_RSSI_dBm.pdf}

\subsection{Mode6}
\printfile{Mode6/results/RX_RSSI_dBm_gps.jpg}
\printfile{Mode6/results/RX_RSSI_dBm.pdf}

\printfile{Mode6/results/UL_RSSI_dBm_gps.jpg}
\printfile{Mode6/results/UL_RSSI_dBm.pdf}

\subsection{Path loss}
\printfile{results/RX_RSSI_dBm_dist_bars.pdf}
\printfile{results/RX_RSSI_dBm_dist_with_PL.pdf}


\section{PBCH comparison}
We show the frame error rate on the PBCH in mode 1 and 2. A more detailed comparison of some interresing parts of the routes will be given in section \ref{sec:dist_travelled}.

\printfile{Mode1/results/PBCH_FER.pdf}
\printfile{Mode1/results/PBCH_fer_gps.jpg}

\printfile{Mode2_update/results/PBCH_FER.pdf}
\printfile{Mode2_update/results/PBCH_fer_gps.jpg}

\section{Rice Factor}

\printfile{Mode2/results/K_factor_gps.jpg}

\section{UE mode comparison}
The following figures show the UE mode of the different measurments. The UE mode is defined in the following table.

\begin{center}
% use packages: array
\begin{tabular}{l|p{6cm}}
UE mode & meaning \\ 
\hline
0 (NOT SYNCHED) & Not synchronized \\ 
1 (PRACH) & UE synchronized to eNB (DL), trying to establish UL using the PRACH\\ 
2 (RAR) & eNB received PRACH, returns RAR \\ 
3 (PUSCH) & UL connection established
\end{tabular}
\end{center}


\printfile{Mode1/results/UE_mode_gps.jpg}
\printfile{Mode2/results/UE_mode_gps.jpg}

\printfile{Mode2_update/results/UE_mode_gps.jpg}
\printfile{Mode6/results/UE_mode_gps.jpg}

\section{L1 (coded) Throughput}


\subsection{Introduction}


In the LTE specification of 5MHz there are 25 Physical Resource Blocks (PRB). An extended cyclic prefix is used making the resource elements(RE) equal to 144 in one downlink subframe of one PRB and total of 3600 REs for 25 PRBs. Out of these 144 REs of one PRB 36 are used for control signals and another 12 are used as cell specific reference signals so the effective REs for data transmission in one PRB in subframe are then, 144-36-12 = 96. Since there are 25 PRBs so the total number of available downlink REs for DLSCH in one subframe become then 25 * 96 = 2400. This results in a maximum downlink throughput of 2.88 Mbps using QPSK, 5.76 Mbps using 16Qam and 8.64 Mbps using 64Qam. Similarly each uplink subframe has 3600 RE, out of which 600 are used for pilots and 300 for the SRS, leving 2700 REs for the ULSCH. This results in maximum uplink throughput of 1.62 Mbps using QPSK, 3.24 Mbps using 16Qam and 4.86 Mbps using 64Qam. 

\subsubsection{Ideal Throughput Calculation}
The channel estimation is done through 200 cell specific reference signals making 200/25 = 8 per receive antenna in each PRB in each subframe. So in each PRB 8 channel estimates are obtained and for these 8 channel estimates, capcity per resource element is calculated using the following capacity formula of finite constellation size $Q_{m}$,

 
% \begin{equation} \label{eq:capacity}
% C(\underline{\textbf{h}},N) = \max_{Q_{m}=\{2,4,6\}} \sum_{i = 0}^{N_{RE}-1}\left[ Q_{m}-\frac{1}{Q_{m}}\sum_{m = 0}^{Q_{m}-1}\int_{-\inf}^{\inf} \! p(y_{i}/h_{i},m)\log_{2}\frac{p(y_{i}/h_{i},m)}{p(y_{i})} \, dy_i\right] 
% \end{equation}

\begin{align}
C_m\left(N_0\right)
&=\log M-\frac{1}{MN_{z}N_{H}}\sum_{x_{1}\in{Q_m}}\sum_{h_1}^{N_{h}}\sum_{z_{1}}^{N_{z}}\log\frac{\sum_{x^{'}_{1}}\exp\left[-\frac{1}{N_{0}}\left|y_{1}-h_{1}x^{'}_{1}\right|^2\right]}{\exp\left[-\frac{1}{N_{0}}\left|y_{1}-h_{1}x_{1}\right|^{2}\right]}\nonumber\\
&=\log M-\frac{1}{MN_{z}N_{H}}\sum_{x_{1}\in{Q_m}}\sum_{h_1}^{N_{h}}\sum_{z_{1}}^{N_{z}}\log\frac{\sum_{x^{'}_{1}}\exp\left[-\frac{1}{N_{0}}\left|h_{1}x_{1} + z_1 - h_1x^{'}_{1}\right|^2\right]}{\exp\left[-\frac{1}{N_{0}}\left|z_{1}\right|^{2}\right]}\nonumber\\
\label{eq:capacity}
\end{align}

Where $y_1 = h_1x_1 + z_1$ is the received signal at the receiver. (\ref{eq:capacity}) is the formula for ergodic capacity of random fading channel per resource element and is a function of SNR and supported modulation scheme. At first the SNR for all the three transmission modes is calculated and then using Shanon capacity formula the supported modulation scheme is identified. After that the capacity of each RE is calculated using (\ref{eq:capacity}), SNR and supported modulation scheme. The SNR for transmission mode 1,2 and 6 is calculated in the following way:

\begin{equation} \label{eq:snrsiso}
SNR_{mode1_{2Rx}} = 10*{\mathrm{log}}_{10} \left\lbrace {\sum_{i=1}^2} {\frac{{\| \ {h_{1i}} + {h_{2i}} \| \\}^2}{N_{0i}}}\right\rbrace
\end{equation}

Since antenna configuration during measurement campaign is 2x2 on downlink and in transmission mode 1 the signal is replicated on both of the transmit antennas so superposition of both channels is considered at the each receive antenna and then Maximal Ratio Combining (MRC) is applied at receiver to reach (\ref{eq:snrsiso}). 
 
\begin{equation} \label{eq:snralam}
SNR_{mode2_{2Rx}} = 10* {\mathrm{log}}_{10} \left\lbrace {\sum_{i=1}^2} \, {\frac{ {\| \ {h_{1i}} \| \\}^2 +  {\| \ {h_{2i}} \| \\}^2 }{N_{0i}}}\right\rbrace. 
\end{equation}

In transmission mode 2 , two complex symbols (i.e. $s_1$ and $s_2$) are transmitted over two symbol times from two Tx antennas. In first symobl time $s_1$ and $s_2$ are transmitted through antenna 1 and antenna 2 respectively whereas in second symbol time $-{s_{2}^{*}}$ and ${s_{1}^{*}}$ is transmitted through antenna 1 and antenna 2 respectively. This gives diversity order of 2 at each of the receive antenna where both of the received channels are considered disticntively thus giving rise in received SNR. 

\begin{equation} \label{eq:snrbmfr}
SNR_{mode6_{2Rx}} = 10* {\mathrm{log}}_{10} \left\lbrace {\sum_{i=1}^2} {\frac{{\| \ {h_{1i}} + q*{h_{2i}} \| \\}^2}{N_{0i}}}\right\rbrace. 
\end{equation}

And in transmission mode 6 high SNR is obtained by using precoder $q$ which focuses the transmit energy in specific direction only, so the joint precoded channel is considered in (\ref{eq:snrbmfr}). In ideal capacity calculation for transmission mode 6, two methods of precoder calculation are considered. In first method the feedback from UE is utilized and sum capacity is calcualted, where as in the second method the optimal $q$ which maximizes the overall sum capacity is calculated. Please note that $q$ is selected on subband basis in each subframe.


% In (\ref{eq:snrsiso}), (\ref{eq:snralam}) and (\ref{eq:snrbmfr}) the $N_{i}$, is the noise variance on each of the receive antennas with $ i = \left\lbrace 1,2\right\rbrace $. 
After calculating SNR for each of the transmission mode, Shannon capacity for each RE is calculated using the famous Shannon Capacity formula $C = \log_2(1 + SNR)$ to see what modulation scheme be supported for this RE. Then using this supported modulation scheme and (\ref{eq:capacity}) the ideal capacity for each RE is calculated.
\newline
Then capacity of rest of 96 REs is extrapolated by multiplying the sum capacity of 8 channel estimates per receive antenna by the factor of 12 which which gives us the sum capacity of one PRB for one subframe. The same procedure is done for the rest of 24 PRBs and at the end their cumulative sum is taken giving us the downlink throughput of one complete subframe. Since Uplink-downlink configuration 3 is used in which there are 6 downlink subframes in a frame so then the sum capacity of one subframe is multiplied by the factor of 6 to calculate the downlink throughput of one frame. 

One frame in LTE is of 1ms duration. In order to get the downlink throughput per second, the downlink throughput of 100 frames is added together giving the throughput in bits per seconds.

In ideal capacity calculation the downlink capacity is measured for the single antenna and two antennas UE.  

For the ideal curves we use the channel measurments from mode 2.

\subsubsection{Modem Throughput Calculation}



\subsection{CDF}

% \printfile{Mode1/results/DLSCH_throughput_cdf_comparison.pdf}
% \printfile{Mode2/results/DLSCH_throughput_cdf_comparison.pdf}
% \printfile{Mode6/results/DLSCH_throughput_cdf_comparison.pdf}

We plot the CDF of the throughput that was measured with our modem as well as the ideal throughputs as explained in the Introduction. In the first plot the data includes all the measurment points, even when the UE was not connected (in which case the throughput is 0). Note that data for the ideal curves is the same as for mode 2. 

However, since the routes of the measurements for mode 1 and mode 6 were not exactly the same as for mode 2 (measurments have not be started and stopped at the same points, routes were taken in different orders, crashes of the equipment, \ldots), a comparison of the outage between the different modes is unfair. Therefore in a second comparison we only compare the points where the UE was connected. 

Last but not least we show the throughput of the UL, where we have only used the measurements of mode 6.

\printfile{results/DLSCH_throughput_cdf_comparison.pdf}
\printfile{results/DLSCH_throughput_connected_cdf_comparison.pdf}

\printfile{Mode6/results/UL_throughput_cdf_comparison.pdf}


\subsection{Time}

\printfile{Mode1/results/DLSCH_throughput.pdf}
\printfile{Mode2/results/DLSCH_throughput.pdf}

\printfile{Mode6/results/DLSCH_throughput.pdf}
\printfile{Mode2/results/coded_throughput_time_1stRx.pdf}

\printfile{Mode2/results/coded_throughput_time_2ndRx.pdf}
\printfile{Mode2/results/coded_throughput_time_2Rx.pdf}

\printfile{Mode6/results/UL_throughput.pdf}


\subsection{Map}

\printfile{Mode1/results/DLSCH_troughput_gps.jpg}
\printfile{Mode2/results/DLSCH_troughput_gps.jpg}

\printfile{Mode6/results/DLSCH_troughput_gps.jpg}
\printfile{Mode6/results/UL_throughput_gps.jpg}

\printfile{Mode2/results/coded_throughput_Alamouti_gps_2Rx.jpg}
\printfile{Mode2/results/coded_throughput_SISO_gps_2Rx.jpg}

\printfile{Mode2/results/coded_throughput_feedbackBeamforming_gps_2Rx.jpg}
\printfile{Mode2/results/coded_throughput_optBeamforming_gps_2Rx.jpg}

\printfile{Mode6/results/UL_throughput_ideal_1Rx_gps.jpg}
\printfile{Mode6/results/UL_throughput_ideal_2Rx_gps.jpg}

\subsection{Distance}

\printfile{Mode1/results/DLSCH_throughput_dist.pdf}
\printfile{Mode2/results/DLSCH_throughput_dist.pdf}

\printfile{Mode6/results/DLSCH_throughput_dist.pdf}
\printfile{Mode6/results/UL_throughput_dist.pdf}

%\printfile{Mode2/results/ideal_throughput_dist_mode1_1stRx.pdf}
%\printfile{Mode2/results/ideal_throughput_dist_mode1_2ndRx.pdf}

\printfile{Mode2/results/ideal_throughput_dist_mode1_2Rx.pdf}
%\printfile{Mode2/results/ideal_throughput_dist_mode2_1stRx.pdf}
%
%\printfile{Mode2/results/ideal_throughput_dist_mode2_2ndRx.pdf}
\printfile{Mode2/results/ideal_throughput_dist_mode2_2Rx.pdf}

%\printfile{Mode2/results/ideal_throughput_dist_mode6_feedbackq_1stRx.pdf}
\printfile{Mode2/results/ideal_throughput_dist_mode6_feedbackq_2Rx.pdf}
%
%\printfile{Mode2/results/ideal_throughput_dist_mode6_maxq_1stRx.pdf}
\printfile{Mode2/results/ideal_throughput_dist_mode6_maxq_2Rx.pdf}

\printfile{Mode6/results/UL_throughput_ideal_1Rx_dist.pdf}
\printfile{Mode6/results/UL_throughput_ideal_2Rx_dist.pdf}

\begin{table}
\centering
\begin{tabular}{l|l|l|l}
Dist (km) & Mode 1 & Mode 2 & Mode 6\\
\hline
0--1 &   0.9383 &   0.6966 &   0.9771\\
1--2 &   0.9827 &   0.9318 &   0.9701\\
2--3 &   0.8842 &   0.7704 &   0.8900\\
3--4 &   0.7433 &   0.6468 &   0.6591\\
4--5 &   0.7743 &   0.7886 &   0.7733\\
5--6 &   0.5794 &   0.5523 &   0.5955\\
6--7 &   0.3655 &   0.4885 &   0.5915\\
7--8 &   0.5126 &   0.5354 &   0.7587\\
8--9 &   0.5105 &   0.5339 &   0.5628\\
9--10 &    0.3728 &   0.3897 &   0.2421\\
10--11 &   0.4400 &   0.3487 &   0.4488\\
11--12 &   0.4279 &   0.4098 &   0.4582\\
12--13 &   0.2374 &   0.2106 &   0.2939\\
13--14 &   0.1625 &   0.2037 &   0.1122\\
14--15 &   0.2809 &   0.1788 &   0.2536\\
15--16 &   0.1053 &   0.1839 &   0.0649\\
16--17 &   0.3062 &   0.2259 &   0.4022\\
\end{tabular}
\caption{Service coverage for all three modes in percent.}
\end{table}

\subsection{Distance travelled}
\label{sec:dist_travelled}
For a closer comparison of the 

For the comparsion over distance travelled we select a small but representative portion of the measurements shown in the next figure.

TODO: add PBCH FER for zoom 1 and 2 (this requires to select the correspondig portion of the mode2 update); plot CQI and CQI gain


\printfile[Map of zoom 1]{results/all_modes_comparison_rssi_dBm.jpg}
\printfile[RSSI comparison of zoom 1]{results/all_modes_comparison_rssi_dBm.pdf}

\printfile[DLSCH throughput of zoom 1]{results/all_modes_comparison_troughput_distance_travelled.pdf}
\printfile[DLSCH CQI of zoom 1]{results/all_modes_comparison_cqi_distance_travelled.pdf}

\printfile[Map of zoom 2]{results/all_modes_comparison2_rssi_dBm.jpg}
\printfile[RSSI comparison of zoom 2]{results/all_modes_comparison2_rssi_dBm.pdf}

\printfile[DLSCH throughput of zoom 2]{results/all_modes_comparison2_troughput_distance_travelled.pdf}
\printfile[DLSCH CQI of zoom 2]{results/all_modes_comparison2_cqi_distance_travelled.pdf}

\printfile[Map of zoom 3]{results/all_modes_comparison3_rssi_dBm.jpg}
\printfile[RSSI comparison of zoom 3]{results/all_modes_comparison3_rssi_dBm.pdf}

\printfile[DLSCH throughput of zoom 3]{results/all_modes_comparison3_troughput_distance_travelled.pdf}
\printfile[PBCH FER of zoom 3]{results/all_modes_comparison3_pbch_fer_distance_travelled.pdf}

\subsection{Speed}

\printfile{Mode1/results/DLSCH_throughput_speed.pdf}
\printfile{Mode2/results/DLSCH_throughput_speed.pdf}

\printfile{Mode6/results/DLSCH_throughput_speed.pdf}

\printfile{Mode6/results/UL_throughput_ideal_1Rx_gps.jpg}
\printfile{Mode6/results/UL_throughput_ideal_2Rx_gps.jpg}

\subsubsection{Distance}
\printfile{Mode6/results/UL_throughput_dist.pdf}
\printfile{Mode6/results/UL_throughput_ideal_1Rx_dist.pdf}
\printfile{Mode6/results/UL_throughput_ideal_2Rx_dist.pdf}

\subsubsection{Speed}
\printfile{Mode6/results/UL_throughput_speed.pdf}

\begin{table}
\centering
\begin{tabular}{l|l|l|l}
speed (m/s) & Mode 1 & Mode 2 & Mode 6\\
\hline
0--5   &   0.3876  &  0.5360 &   0.8385\\
5--10  &   0.6776  &  0.4844 &   0.6915\\
10--15 &   0.5553  &  0.5015 &   0.6192\\
15--20 &   0.5694  &  0.5402 &   0.6677\\
20--25 &   0.5133  &  0.5376 &   0.5087\\
25--30 &   0.5364  &  0.5480 &   0.6111\\
30--35 &   0.0909  &     NaN &      NaN\\
35--40 &      NaN  &     NaN &      NaN\\
\end{tabular}
\caption{Service coverage for all three modes in percent.}
\end{table}

\section{L0 (uncoded) throughput}

\printfile{results/DLSCH_uncoded_throughput_cdf_comparison.pdf}
\printfile{results/DLSCH_uncoded_throughput_connected_cdf_comparison.pdf}

\printfile{Mode6/results/UL_uncoded_throughput_cdf_comparison.pdf}

\end{document}          
