\documentclass[a4paper,10pt]{article}

\usepackage[pdftex,dvips]{graphicx}
\usepackage[T1]{fontenc}
\usepackage{url}
\usepackage[top=2cm, bottom=2cm, left=2cm, right=2cm]{geometry}
\usepackage{amsmath}

\newcommand{\printfile}[2][]{
 \begin{minipage}{8cm}
  \centering
  \includegraphics[width=8cm]{/extras/kaltenbe/CNES/emos_postprocessed_data/#2}
  %\includegraphics[width=8cm]{/emos/EMOS/#2}
  \url{#2}: #1

 \end{minipage}
}

\title{Post Processing Results of LTE Measurement Campaign}
\author{Florian Kaltenberger, Raymond Knopp, Imran Latif, Rizwan Ghaffar,\\ 
Dominique Nussbaum, Herve Callewaert} 

%\addtolength{\textwidth}{3cm}
%\setlength{\marginparwidth}{0cm}
%\setlength{\hoffset}{0cm}

\begin{document}

\maketitle

\tableofcontents

\pagebreak

% \subsection{Mode 1}
% 
% \printfile{./Mode1/20100526_mode1_parcours1_part4_part5/RX_RSSI_dBm_gps.jpg}
% \printfile{./Mode1/20100527_mode1_parcours1/RX_RSSI_dBm_gps.jpg}
% 
% \subsection{Mode 2}
% \printfile{./Mode2/20100510_mode2_parcours1_part1/RX_RSSI_dBm_gps.jpg}
% \printfile{./Mode2/20100510_mode2_parcours1_part2/RX_RSSI_dBm_gps.jpg}
% 
% \printfile{./Mode2/20100510_mode2_parcours1_part3.1/RX_RSSI_dBm_gps.jpg}
% \printfile{./Mode2/20100510_mode2_parcours1_part3.2/RX_RSSI_dBm_gps.jpg}
% 
% \printfile{./Mode2/20100511_mode2_parcours1_part4_5_6/RX_RSSI_dBm_gps.jpg}
% \printfile{./Mode2/20100511_mode2_parcours2_part5/RX_RSSI_dBm_gps.jpg}
% 
% \printfile{./Mode2/20100512_mode2_parcours2_part7/RX_RSSI_dBm_gps.jpg}
% %\printfile{./Mode2/20100512_mode2_parcours2_part7/nomadic/RX_RSSI_dBm_gps.jpg}
% \printfile{./Mode2/20100518_mode2_parcours1_part6.1/RX_RSSI_dBm_gps.jpg}
% %\printfile{./Mode2/20100518_mode2_parcours1_part6.1/nomadic/RX_RSSI_dBm_gps.jpg}
% 
% \printfile{./Mode2/20100518_mode2_parcours1_part6.2/RX_RSSI_dBm_gps.jpg}
% %\printfile{./Mode2/20100518_mode2_parcours1_part6.2/nomadic/RX_RSSI_dBm_gps.jpg}
% \printfile{./Mode2/20100518_mode2_parcours1_part6.3/RX_RSSI_dBm_gps.jpg}
% %\printfile{./Mode2/20100518_mode2_parcours1_part6.3/nomadic/RX_RSSI_dBm_gps.jpg}
% 
% \printfile{./Mode2/20100519_mode2_parcours1_part7.1/RX_RSSI_dBm_gps.jpg}
% \printfile{./Mode2/20100519_mode2_parcours1_part7.2/RX_RSSI_dBm_gps.jpg}
% %\printfile{./Mode2/20100519_mode2_parcours1_part7.2/nomadic/RX_RSSI_dBm_gps.jpg}
% 
% \printfile{./Mode2/20100519_mode2_parcours1_part7.3/RX_RSSI_dBm_gps.jpg}
% %\printfile{./Mode2/20100519_mode2_parcours1_part7.3/nomadic/RX_RSSI_dBm_gps.jpg}
% \printfile{./Mode2/20100520_mode2_parcours1_part4_part5/RX_RSSI_dBm_gps.jpg}
% 
% \printfile{./Mode2/20100520_mode2_parcours1_part9_parcours2_part1/RX_RSSI_dBm_gps.jpg}
% \printfile{./Mode2/20100521_mode2_parcours2_part1_part2/RX_RSSI_dBm_gps.jpg}
% 
% \printfile{./Mode2/20100524_mode2_parcours2_part3_part4.1/RX_RSSI_dBm_gps.jpg}
% \printfile{./Mode2/20100524_mode2_parcours2_part3_part4.2/RX_RSSI_dBm_gps.jpg}
% 
% \printfile{./Mode2/20100524_mode2_parcours2_part3_part4.3/RX_RSSI_dBm_gps.jpg}
% \printfile{./Mode2/20100525_mode2_parcours2_part4_part8_part9_part10.1/RX_RSSI_dBm_gps.jpg}
% 
% \printfile{./Mode2/20100525_mode2_parcours2_part4_part8_part9_part10.2/RX_RSSI_dBm_gps.jpg}
% \printfile{./Mode2/20100525_mode2_parcours2_part4_part8_part9_part10.3/RX_RSSI_dBm_gps.jpg}
% 
% \printfile{./Mode2/20100607_VTP_MODE2_PARCOURS1_S2_3/RX_RSSI_dBm_gps.jpg}
% \printfile{./Mode2/20100608_VTP_MODE2_PARCOURS1_S7_9/RX_RSSI_dBm_gps.jpg}
% 
% \printfile{./Mode2/20100608_VTP_MODE2_PARCOURS1_S8/RX_RSSI_dBm_gps.jpg}
% 
% \subsection{Mode 6}
% 
% \printfile{./Mode6/20100528_mode6_parcours1_subsection1/RX_RSSI_dBm_gps.jpg}
% \printfile{./Mode6/20100528_mode6_parcours1_subsection2/RX_RSSI_dBm_gps.jpg}
% 
% \printfile{./Mode6/20100608_VTP_MODE6_ZONES_PUSCH_PART1/RX_RSSI_dBm_gps.jpg}
% \printfile{./Mode6/20100608_VTP_MODE6_ZONES_PUSCH_PART2/RX_RSSI_dBm_gps.jpg}
% 
% \printfile{./Mode6/20100608_VTP_MODE6_ZONES_PUSCH_PART3/RX_RSSI_dBm_gps.jpg}
% \printfile{./Mode6/20100609_VTP_MODE6_ZONES_PUSCH_PART4/RX_RSSI_dBm_gps.jpg}
% 
% \printfile{./Mode6/20100610_VTP_MODE6_ZONES_PUSCH_UPDATE.1/RX_RSSI_dBm_gps.jpg}
% \printfile{./Mode6/20100610_VTP_MODE6_ZONES_PUSCH_UPDATE.2/RX_RSSI_dBm_gps.jpg}
% 
% \subsection{UL}
% \printfile{./Mode6/20100608_VTP_MODE6_ZONES_PUSCH_PART3/UL_RSSI_dBm_gps.jpg}
% \printfile{./Mode6/20100608_VTP_MODE6_ZONES_PUSCH_PART2/UL_RSSI_dBm_gps.jpg}
% 
% \printfile{./Mode6/20100608_VTP_MODE6_ZONES_PUSCH_PART1/UL_RSSI_dBm_gps.jpg}
% \printfile{./Mode6/20100609_VTP_MODE6_ZONES_PUSCH_PART4/UL_RSSI_dBm_gps.jpg}
% 
% \printfile{./Mode6/20100610_VTP_MODE6_ZONES_PUSCH_UPDATE.1/UL_RSSI_dBm_gps.jpg}
% \printfile{./Mode6/20100610_VTP_MODE6_ZONES_PUSCH_UPDATE.2/UL_RSSI_dBm_gps.jpg}
% 
% \subsection{UL ideal}
% \printfile{./Mode6/20100608_VTP_MODE6_ZONES_PUSCH_PART3/UL_throughput_ideal_2Rx_gps.jpg}
% \printfile{./Mode6/20100608_VTP_MODE6_ZONES_PUSCH_PART2/UL_throughput_ideal_2Rx_gps.jpg}
% 
% \printfile{./Mode6/20100608_VTP_MODE6_ZONES_PUSCH_PART1/UL_throughput_ideal_2Rx_gps.jpg}
% \printfile{./Mode6/20100609_VTP_MODE6_ZONES_PUSCH_PART4/UL_throughput_ideal_2Rx_gps.jpg}
% 
% \printfile{./Mode6/20100610_VTP_MODE6_ZONES_PUSCH_UPDATE.1/UL_throughput_ideal_2Rx_gps.jpg}
% \printfile{./Mode6/20100610_VTP_MODE6_ZONES_PUSCH_UPDATE.2/UL_throughput_ideal_2Rx_gps.jpg}

%\subsection{all plots}

\section{Introduction}

This document summarizes the results of the LTE measurement campaign that was conducted by Eurecom and Sogeti April-July 2010 for the CNES.

The Eurecom testbench implements the LTE 3GPP release 8.6 \cite{3GPPTS36.211,3GPPTS36.212,3GPPTS36.213} with 5 MHz bandwidth and TDD uplink-downlink configuration 3 (i.e., there are 6 downlink subframes and 3 uplink subframes in a frame of 10ms). Extended cyclic prefix is used in both UL and DL. The carrier frequency is 859.6 Mhz. 
 
The measurement methodology and specification of the post processing are described in \cite{measurements_spec}. The measurment sites are Cordes-sur-ciel \cite{cordes_desc}, Penne \cite{penne_desc}, and Ambialet \cite{ambialet_desc}. Table \ref{tab:meas} lists the modes and their corresponding routes that were taken at each  measurement site (see also the test plan \cite{test_plan}). More specifically we list in table \ref{tab:meas_list} which measurements from the table \cite{measurements_spreadsheet} have been used.


\begin{table}
\centering
% use packages: array
\begin{tabular}{l|p{6cm}}
Type of measurement & Routes taken \\ 
\hline
Interference & exterior road \\ 
Mode1 & parcours 2 \\ 
Mode2 & parcours 1, 2, and exterior road \\ 
Mode6 & parcours 2, zones PUSCH \\ 
\end{tabular}
\caption{List of measurements and routes}
\label{tab:meas}
\end{table}


\begin{table}
\centering
% use packages: array
\begin{tabular}{l|l}
\hline
Type of measurement & Lines in \cite{measurements_spreadsheet} \\ 
\hline
\hline
CORDES\\
\hline
Interference & 6,8,9 \\ 
Mode1 & 42-43 \\ 
Mode2 & 23,24,27--42\\
Mode2 update & 46--47,50--52\footnote{Measurements 50--52 were actually taken in mode 6. This is fine, since these measurements are only used for the evaluation of the PBCH performance and the transmission format of the PBCH is the same in mode 2 and mode 6.}\\ 
Mode6 & 45,48--52 \\ 
\hline
PENNE\\
\hline
Interference & 56,57,60,61 \\ 
Mode1 & 67,69,70 \\ 
Mode2 (OFDMA) &  58,59\\ 
Mode2 &  63,65,66,71--73\\ 
Mode6 &  68\\ 
\hline
AMBIALET\\
\hline
Interference &  80,81\\ 
Mode1 & 82,84--86,96\\ 
Mode2 &  89,92,94\\ 
Mode6 &  91,95\\ 

\end{tabular}
\caption{List of measurements and routes}
\label{tab:meas_list}
\end{table}
 
For the site of Cordes, some mode2 measurements have been taken twice. This is due to a bug\footnote{The receiver for the PBCH was always assuming transmission mode 1. This caused a slight degradation in the performance in mode2, but only at the cell edge (at high SNR, both receiver structures work fine).} that was discovered in the reception of the PBCH in transmission mode 2. This data is only used for the evaluation of the PBCH performance. Also, all measurements were taken using OFDMA in the uplink. For the site of Penne, the uplink was switched to SC-FDMA, but some mode 2 measurements using OFDMA were also conducted. For the site of Ambialet, only SC-FDMA was used in the UL.

The data was recorded with a frequency of 100Hz, i.e., one measurement every 10ms (= 1 LTE frame). The GPS data was recorded with a frequency of 1Hz, i.e., one measurement every second. Most of the data in this report thus shows one measurement point per second. We will specify in the following subsections how we generate this data. The structure of the subsections is maintained in the subsequent sections, which show the results for the different sites. 

\subsection{Interference measurements}

After setting up the eNB and UE at a specific cell site, interference measurement (IM) is the very first and most important step of measurement campaign. Interference measurements are conducted to make sure that under test frequency band is exclusively used for the campaign and no other operator is using this frequency band in a specific cell. The two figures in this section show the measured interference in dBm at the UE and at the eNB. 

For the interference measurements at the UE, the UE was put into standby mode. In this mode the RX RSSI is measured based on a snapshot of 10ms and is computed every 50ms. The signal strenghts of the two receive antennas are summed up. For the visuzlization we apply a decimation factor of 100, i.e., we plot one interference measurement every 5 seconds.

For the interference measurements at the eNB, the eNB was put in normal transmission mode, but without connecting the transmit RF chain. In this mode the RX RSSI is measured based on the received signal in the RX part of the special subframe. The signal strenghts of the two receive antennas are summed up. That means we get one measurement every 10 ms, which is averaged using a moving average filter. For the visualization we apply a decimation factor of 100, i.e., we plot one interference measurement every second. 

\subsection{RX RSSI}

The calculation of the received signal strength indicator (RSSI) depends on the UE mode. If the UE is not synched to the BS, the RSSI is calculated the same way as it was done in the interference measurement. The signal strenghts of the two receive antennas are summed up. If the UE is synched to the eNB, the RSSI calculation is based on the channel measurements. Again, the signal strenghts of the two receive antennas are summed up. For the visualization we apply a decimation factor of 100, i.e., we plot one RX RSSI measurement every 5 secods or every second, depending on the UE mode.

The RSSI of both UE and eNB measured is plotted on the regional map of the specific cell as well as a function of time. We show the RSSI in dBm for the coverage run, the measurements in mode 1, 2, and 6 as well as the uplink RSSI. 

Further we show the UE RSSI measurements of all measurements as a function of the distance. The plot also shows a fitted simple path loss model $\mathrm{PL}(R) = \mathrm{PL}_0 + 10n\log_{10}(R)$ as well as the path loss according to the COST231-Hata model \cite{cost231}.

\subsection{PBCH comparison}
We show the frame error rate on the PBCH in mode 1 and 2. A more detailed comparison of some interresing parts of the routes will be given in subsection \ref{sec:dist_travelled}

\subsection{Rice Factor}
This plot shows the Rice factor on a map. The Rice factor has been calculated using the method of moments \cite{Greenstein1999}, averaging over widband channel estimates of 100 frames. We therfore get one Rice factor estimate every second. 

\subsection{UE mode comparison}
The figures in this section show the UE mode of the different measurements with a decimation factor of 100 (i.e., one point every second). The UE mode is defined in the following table.

\begin{center}
% use packages: array
\begin{tabular}{l|p{6cm}}
UE mode & meaning \\ 
\hline
0 (NOT SYNCHED) & Not synchronized \\ 
1 (PRACH) & UE synchronized to eNB (DL), trying to establish UL using the PRACH\\ 
2 (RAR) & eNB received PRACH, returns RAR \\ 
3 (PUSCH) & UL connection established
\end{tabular}
\end{center}


\subsection{L1 (coded) Throughput}

In the LTE specification of 5MHz there are 25 Physical Resource Blocks (PRB). In the extended cyclic prefix configuration each PRB consists of 144 resource elements (RE), giving a total of 3600 REs for 25 PRBs. On the downlink, out of the 144 REs of one PRB 32 are used for control signals and another 16 are used as cell specific reference signals (in the two transmit antenna configuration) so the effective REs for data transmission in one PRB are 144-32-16 = 96. Since there are 25 PRBs so the total number of available downlink REs for DLSCH in one subframe become then 25 * 96 = 2400. This results in a maximum downlink throughput of 2.88 Mbps using QPSK, 5.76 Mbps using 16Qam and 8.64 Mbps using 64Qam. Similarly each uplink subframe has 3600 RE, out of which 600 are used for pilots and 300 for the SRS, leaving 2700 REs for the ULSCH. This results in maximum uplink throughput of 1.62 Mbps using QPSK, 3.24 Mbps using 16Qam and 4.86 Mbps using 64Qam. Please note that these throughputs for both DL and UL are uncoded throughputs. 

\subsubsection{Ideal Throughput Calculation}
The calculation of the ideal throughput follows loosely the method described in \cite{IEEE802.16m_EMD}. For the ideal curves we use the channel measurements from mode 2. The channel estimation results in 200 channel estimates $h_{j,i}$ for each subframe and each transmit-receive antenna pair $(j,i)$. To calculate the ideal throughput for such a subframe we apply the following 3 steps.
\begin{enumerate}
 \item Calculate the effective SNR based on the transmission mode and the feedback for each channel estimate in a subframe
 \item Calculate the best supported modulation scheme for this subframe
 \item Calculate the mutual information of the subframe under the constraint of the supported modulation scheme
 \item Scale the results down to include  the effects of channel estimation, decoding performance, etc. 
\end{enumerate}

\paragraph{Calculation of the effective SNR.}
The SNR for transmission mode 1,2 and 6 is calculated in the following way:

\begin{equation} \label{eq:snrsiso}
\mathrm{SNR}_{\mathrm{mode1}} = 10\cdot{\mathrm{log}}_{10} \left\lbrace {\sum_{i=1}^{n_{\mathrm{Rx}}}} {\frac{{\| \ {h_{1i}} + {h_{2i}} \| \\}^2}{N_{0i}}}\right\rbrace,
\end{equation}
where $N_{0i}$ is the noise variance of receive antenna $i$ and $n_{\mathrm{Rx}}$ is the number of RX antennas.

Since antenna configuration during measurement campaign is 2x2 on downlink and in transmission mode 1 the signal is replicated on both of the transmit antennas so superposition of both channels is considered at the each receive antenna and then Maximal Ratio Combining (MRC) is applied at receiver to reach (\ref{eq:snrsiso}). 
 
\begin{equation} \label{eq:snralam}
\mathrm{SNR}_{\mathrm{mode2}} = 10\cdot{\mathrm{log}}_{10} \left\lbrace {\sum_{i=1}^{n_{\mathrm{Rx}}}} \, {\frac{ {\| \ {h_{1i}} \| \\}^2 +  {\| \ {h_{2i}} \| \\}^2 }{N_{0i}}}\right\rbrace. 
\end{equation}

In transmission mode 2 , two complex symbols (i.e. $s_1$ and $s_2$) are transmitted over two symbol times from two Tx antennas. In first symbol time $s_1$ and $s_2$ are transmitted through antenna 1 and antenna 2 respectively whereas in second symbol time $-{s_{2}^{*}}$ and ${s_{1}^{*}}$ is transmitted through antenna 1 and antenna 2 respectively. This gives diversity order of 2 at each of the receive antenna where both of the received channels are considered disticntively thus giving rise in received SNR. 

\begin{equation} \label{eq:snrbmfr}
\mathrm{SNR}_{\mathrm{mode6}} = 10\cdot{\mathrm{log}}_{10} \left\lbrace {\sum_{i=1}^{n_{\mathrm{Rx}}}} {\frac{{\| \ {h_{1i}} + q*{h_{2i}} \| \\}^2}{N_{0i}}}\right\rbrace. 
\end{equation}

And in transmission mode 6 high SNR is obtained by using precoder $q$ which focuses the transmit energy in specific direction only, so the joint precoded channel is considered in (\ref{eq:snrbmfr}). In ideal capacity calculation for transmission mode 6, two methods of precoder calculation are considered. In first method the feedback from UE is utilized and sum capacity is calcualted, where as in the second method the optimal $q$ which maximizes the overall sum capacity is calculated. Please note that $q$ is selected on subband basis in each subframe.

\paragraph{Calculation of the supported modulation scheme.}
% In (\ref{eq:snrsiso}), (\ref{eq:snralam}) and (\ref{eq:snrbmfr}) the $N_{i}$, is the noise variance on each of the receive antennas with $ i = \left\lbrace 1,2\right\rbrace $. 
After calculating SNR for each of the transmission mode, Shannon capacity for each RE is calculated using the Shannon Capacity formula $$C = \log_2(1 + \mathrm{SNR})$$ and average it over all the REs. We then chose the next possible modulation order (out of QPSK, 16QAM, and 64QAM) above that value  to see what modulation scheme be supported. 

\paragraph{Constrained capacity calculation.}
For each channel estimate, the capacity is calculated using the following capacity formula of finite constellation size $m \in {2,4,6}$ (QPSK, 16QAM, 64QAM) and noise variance $N_0$.

\begin{align}
C_m\left(N_0\right)
&=\log M-\frac{1}{MN_{z}N_{H}}\sum_{x_{1}\in{Q_m}}\sum_{h_1}^{N_{h}}\sum_{z_{1}}^{N_{z}}\log\frac{\sum_{x^{'}_{1}}\exp\left[-\frac{1}{N_{0}}\left|y_{1}-h_{1}x^{'}_{1}\right|^2\right]}{\exp\left[-\frac{1}{N_{0}}\left|y_{1}-h_{1}x_{1}\right|^{2}\right]}\nonumber\\
&=\log M-\frac{1}{MN_{z}N_{H}}\sum_{x_{1}\in{Q_m}}\sum_{h_1}^{N_{h}}\sum_{z_{1}}^{N_{z}}\log\frac{\sum_{x^{'}_{1}}\exp\left[-\frac{1}{N_{0}}\left|h_{1}x_{1} + z_1 - h_1x^{'}_{1}\right|^2\right]}{\exp\left[-\frac{1}{N_{0}}\left|z_{1}\right|^{2}\right]}\nonumber\\
\label{eq:capacity}
\end{align}

Where $y_1 = h_1x_1 + z_1$ is the received signal at the receiver, $h_1$ is a random channel with $\|h_1\|^2=1$, $z_1$ is a circular symmetric Gaussian noise, $Q_m$ is the set of constellation points for the modulation order $m$, and $n_{Rx}$ is the number of receive antennas. (\ref{eq:capacity}) is the formula for ergodic capacity of random fading channel per resource element and is a function of SNR and supported modulation scheme. 


Then capacity of rest of 96 REs is extrapolated by multiplying the sum capacity of 8 channel estimates per receive antenna by the factor of 12 which which gives us the sum capacity of one PRB for one subframe. The same procedure is done for the rest of 24 PRBs and at the end their cumulative sum is taken giving us the downlink throughput of one complete subframe. Since Uplink-downlink configuration 3 is used in which there are 6 downlink subframes in a frame so then the sum capacity of one subframe is multiplied by the factor of 6 to calculate the downlink throughput of one frame. One frame in LTE is of 1ms duration. In order to get the downlink throughput per second, the downlink throughput of 100 frames is added together giving the throughput in bits per seconds.

\paragraph{Scaling.}
The ideal throughput curves are really an upper bound to what the most advanced implementation of an LTE modem can achieve. In the above formulas the following effects are completely neglected:
\begin{itemize}
 \item Channel estimation and interpolation in time and frequency
 \item Decoding performance
 \item All effects of the RF front end.
 \item The formulas assume a perfect rate adaptation and a perfect feedback loop
\end{itemize}

In order to compensate for some of the effects we have carried out simulations with our modem implementation in an AWGN channel. The simulations include channel estimation as well as decoding performance.  Figure \ref{fig:imp_loss} shows the throughput based in mutual information (MI) as well as on simulation results for different SNR values in an AWGN channel. The difference between the MI and the simulated results is called the implementation loss. This implementation loss in applied to all the ideal curves.

\begin{figure}
 \centering
 \includegraphics[width=8cm]{implementation_loss}
 \caption{Throughput based in mutual information (MI) as well as on simulation results for different SNR values in an AWGN channel. The difference between the MI and the simulated results is called the implementation loss.}
 \label{fig:imp_loss}
\end{figure} 

\subsubsection{Modem Throughput Calculation}
To calculate the throughput of the MODEM we sum up the lengths of the successfully received packets within one second. 


\subsubsection{CDF}
\label{sec:cdf}

% \printfile{Mode1/results/DLSCH_throughput_cdf_comparison.pdf}
% \printfile{Mode2/results/DLSCH_throughput_cdf_comparison.pdf}
% \printfile{Mode6/results/DLSCH_throughput_cdf_comparison.pdf}

We plot the CDF of the throughput that was measured with our modem as well as the ideal throughputs as explained in the Introduction. In the first plot the data includes all the measurement points, even when the UE was not connected (in which case the throughput is 0). Note that data for the ideal curves is the same as for mode 2. 

However, since the routes of the measurements for mode 1 and mode 6 were not exactly the same as for mode 2 (measurements have not be started and stopped at the same points, routes were taken in different orders, crashes of the equipment, \ldots), a comparison of the outage between the different modes is unfair. Therefore in a second comparison we only compare the points where the UE was connected. 

In a third comparison we emulate the performance of a broadcast service. This is done by using the ideal throughput measurements for all points where the UE was synched to the eNB. For all other points the throughput is set to 0.

Last but not least we show the throughput of the UL.


\subsubsection{Time}

The plots in this subsection show the DLSCH throughput for mode 1, 2 and 6 over time. Further we show the ideal throughput for the three modes using only the first, only the second or both receive antennas at the UE.


\subsubsection{Map}

The plots in this subsection show the DLSCH throughput for mode 1, 2 and 6 as well as the ideal throughputs for the three modes using  both receive antennas at the UE on a map.

\subsubsection{Distance}

For the plots in this subsection we divide the measurements in bins according to their distance of the UE to the eNb. Each bin is 1km wide and the bin edges are $0, 1, \ldots, \lceil d_{\max} \rceil$, where $d_{\max}$ is the maximum distance in km. For each bin we show the mean, the 5\%, the 50\%, the 85\%, and the 95\% percentile (above) of the DLSCH throughput for mode 1, 2 and 6. 

\subsubsection{Speed}

For the plots in this subsection we divide the measurements in bins according to the UE speed. Each bin is 5m/s wide and the bin edges are $0,5,\ldots,40$ m/s. For each bin we show the mean, the 5\%, the 50\%, the 85\%, and the 95\% percentile (above) of the DLSCH throughput for mode 1, 2 and 6. 

\subsubsection{Ricean factor}

The plot in this subsection shows the throughput as a function of the Ricean factor.

\subsubsection{Extrapolation to loaded cell}

To emulate the extrapolation to a cell with 4 users we split the data set of one measurement into 4 and take each piece as it belonged to a different user. We further assume that the eNB employs a proportionally fair scheduler, that the throughput of every user $T_i$ is scaled by $T_i/T$, where $T=\sum_i T_i$. The sum rate of the eNB is thus
\begin{equation}
 T_{\mathrm{sum}} = \frac{\sum_i T_i^2}{\sum_i T_i}
\end{equation}

\subsubsection{Service Coverage}

The figures in this section show the service coverage in \% as a function of the distance. For the DL it is based on the positive reception of the PBCH and for the UL on the ULSCH. The binning is done in the same way as for the distance plots.

Further we examine the performance of the PBCH closer by plotting the ratio of positive reception as a function of the RX RSSI. The bin edges are given by  $-105, -104, \ldots, -85$ dBm.

\subsubsection{Distance travelled}
\label{sec:dist_travelled}

For a closer comparison of the three modes we select a small but representative subset of the measurements. For each subset we first show the map showing the RX RSSI. Then we show the RX RSSI for all three measurements over the distance travelled. This allows to verify if the measurement routes were the same. Then we show the DLSCH throughput as well as the CQI over the distance travelled for the same route. 

The last subset (Subset 3) has been chosen on the cell edge to allow a comparison of the reception of the PBCH in mode 1 and mode 2.

%TODO: add PBCH FER for zoom 1 and 2 (this requires to select the correspondig portion of the mode2 update); 


\subsection{L0 (uncoded) throughput}

We show the CDF of the uncoded throughput similar to section \ref{sec:cdf}.

\subsection{Nomadic Measurements}

Nomadic measurements were carried out in addition to the vehicular measurements. Each nomadic measurement was taken once with the vehicular antennas and once with the nomadic antennas (reference to datasheet!). In total there were 15 measurement points in Cordes, 10 in Penne and 21 in Ambialet, resulting in a total number of points of 46.

All the results can be found in \cite{nomadic}. In summary it can be seen that the nomadic antennas have a loss of about 4.7dB both in the UL and in the DL. This difference is reflected also in the throughput, where the loss in throughput is 40.996bps for mode 1, 77927 bps for mode 2, 204664bps for mode 6 and 6449 bps for the UL.


\section{Cordes}
\label{sec:cordes}

\subsection{Interference measurements}

\printfile[Interference at eNb]{CORDES/Interference/results/RX_I0_dBm.pdf}
\printfile[Interference at UE]{CORDES/Interference/results/RX_RSSI_dBm_gps.jpg}


\subsection{RX RSSI}


% \subsubsection{Coverage}
% \printfile{CORDES/Coverage/results/RX_RSSI_dBm_gps.jpg}
% \printfile{CORDES/Coverage/results/RX_RSSI_dBm.pdf}
%  
% \printfile{CORDES/Coverage/results/UL_RSSI_dBm_gps.jpg}
% \printfile{CORDES/Coverage/results/UL_RSSI_dBm.pdf}

\subsubsection{Mode 1}
\printfile{CORDES/Mode1/results/RX_RSSI_dBm_gps.jpg}
\printfile{CORDES/Mode1/results/RX_RSSI_dBm.pdf}

\printfile{CORDES/Mode1/results/UL_RSSI_dBm_gps.jpg}
\printfile{CORDES/Mode1/results/UL_RSSI_dBm.pdf}

\subsubsection{Mode2}

\printfile{CORDES/Mode2/results/RX_RSSI_dBm_gps.jpg}
\printfile{CORDES/Mode2/results/RX_RSSI_dBm.pdf}

\printfile{CORDES/Mode2_update/results/UL_RSSI_dBm_gps.jpg}
\printfile{CORDES/Mode2_update/results/UL_RSSI_dBm.pdf}

\subsubsection{Mode6}
\printfile{CORDES/Mode6/results/RX_RSSI_dBm_gps.jpg}
\printfile{CORDES/Mode6/results/RX_RSSI_dBm.pdf}

\printfile{CORDES/Mode6/results/UL_RSSI_dBm_gps.jpg}
\printfile{CORDES/Mode6/results/UL_RSSI_dBm.pdf}

\subsubsection{Path loss}
\printfile{CORDES/results/RX_RSSI_dBm_dist_bars.pdf}
\printfile{CORDES/results/RX_RSSI_dBm_dist_with_PL.pdf}


\subsection{PBCH comparison}
We show the frame error rate on the PBCH in mode 1 and 2. A more detailed comparison of some interresing parts of the routes will be given in subsection \ref{sec:dist_travelled_cordes}.

\printfile{CORDES/Mode1/results/PBCH_FER.pdf}
\printfile{CORDES/Mode1/results/PBCH_fer_gps.jpg}

\printfile{CORDES/Mode2_update/results/PBCH_FER.pdf}
\printfile{CORDES/Mode2_update/results/PBCH_fer_gps.jpg}

\subsection{Rice Factor}

\printfile{CORDES/Mode2/results/K_factor_gps.jpg}

\subsection{UE mode comparison}

\printfile{CORDES/Mode1/results/UE_mode_gps.jpg}
\printfile{CORDES/Mode2/results/UE_mode_gps.jpg}

\printfile{CORDES/Mode2_update/results/UE_mode_gps.jpg}
\printfile{CORDES/Mode6/results/UE_mode_gps.jpg}

\subsection{L1 (coded) Throughput}


\subsubsection{CDF}

% \printfile{Mode1/results/DLSCH_throughput_cdf_comparison.pdf}
% \printfile{Mode2/results/DLSCH_throughput_cdf_comparison.pdf}
% \printfile{Mode6/results/DLSCH_throughput_cdf_comparison.pdf}

\printfile{CORDES/results/throughput_cdf_comparison.pdf}
\printfile{CORDES/results/throughput_connected_cdf_comparison.pdf}

\printfile{CORDES/results/throughput_cdf_comparison_fdd.pdf}

\printfile{CORDES/results/UL_throughput_cdf_comparison.pdf}
\printfile{CORDES/results/UL_throughput_cdf_comparison_fdd.pdf}

\subsubsection{Time}

\printfile{CORDES/Mode1/results/DLSCH_throughput.pdf}
\printfile{CORDES/Mode2/results/DLSCH_throughput.pdf}

\printfile{CORDES/Mode6/results/DLSCH_throughput.pdf}
\printfile{CORDES/Mode2/results/coded_throughput_time_1stRx.pdf}

\printfile{CORDES/Mode2/results/coded_throughput_time_2ndRx.pdf}
\printfile{CORDES/Mode2/results/coded_throughput_time_2Rx.pdf}

\printfile{CORDES/Mode6/results/UL_throughput.pdf}


\subsubsection{Map}

\printfile{CORDES/Mode1/results/DLSCH_troughput_gps.jpg}
\printfile{CORDES/Mode2/results/DLSCH_troughput_gps.jpg}

\printfile{CORDES/Mode6/results/DLSCH_troughput_gps.jpg}
\printfile{CORDES/Mode6/results/UL_throughput_gps.jpg}

\printfile{CORDES/Mode2/results/coded_throughput_Alamouti_gps_2Rx.jpg}
\printfile{CORDES/Mode2/results/coded_throughput_SISO_gps_2Rx.jpg}

\printfile{CORDES/Mode2/results/coded_throughput_feedbackBeamforming_gps_2Rx.jpg}
\printfile{CORDES/Mode2/results/coded_throughput_optBeamforming_gps_2Rx.jpg}

\printfile{CORDES/Mode6/results/UL_throughput_ideal_1Rx_gps.jpg}
\printfile{CORDES/Mode6/results/UL_throughput_ideal_2Rx_gps.jpg}

\subsubsection{Distance}

\printfile{CORDES/Mode1/results/DLSCH_throughput_dist.pdf}
\printfile{CORDES/Mode2/results/DLSCH_throughput_dist.pdf}

\printfile{CORDES/Mode6/results/DLSCH_throughput_dist.pdf}
\printfile{CORDES/Mode6/results/UL_throughput_dist.pdf}

%\printfile{Mode2/results/ideal_throughput_dist_mode1_1stRx.pdf}
%\printfile{Mode2/results/ideal_throughput_dist_mode1_2ndRx.pdf}

\printfile{CORDES/Mode2/results/ideal_throughput_dist_mode1_2Rx.pdf}
%\printfile{Mode2/results/ideal_throughput_dist_mode2_1stRx.pdf}
%
%\printfile{Mode2/results/ideal_throughput_dist_mode2_2ndRx.pdf}
\printfile{CORDES/Mode2/results/ideal_throughput_dist_mode2_2Rx.pdf}

%\printfile{Mode2/results/ideal_throughput_dist_mode6_feedbackq_1stRx.pdf}
\printfile{CORDES/Mode2/results/ideal_throughput_dist_mode6_feedbackq_2Rx.pdf}
%
%\printfile{Mode2/results/ideal_throughput_dist_mode6_maxq_1stRx.pdf}
\printfile{CORDES/Mode2/results/ideal_throughput_dist_mode6_maxq_2Rx.pdf}

\printfile{CORDES/Mode6/results/UL_throughput_ideal_1Rx_dist.pdf}
\printfile{CORDES/Mode6/results/UL_throughput_ideal_2Rx_dist.pdf}

% \begin{table}
% \centering
% \begin{tabular}{l|l|l|l}
% Dist (km) & Mode 1 & Mode 2 & Mode 6\\
% \hline
% 0--1 &   0.9383 &   0.6966 &   0.9771\\
% 1--2 &   0.9827 &   0.9318 &   0.9701\\
% 2--3 &   0.8842 &   0.7704 &   0.8900\\
% 3--4 &   0.7433 &   0.6468 &   0.6591\\
% 4--5 &   0.7743 &   0.7886 &   0.7733\\
% 5--6 &   0.5794 &   0.5523 &   0.5955\\
% 6--7 &   0.3655 &   0.4885 &   0.5915\\
% 7--8 &   0.5126 &   0.5354 &   0.7587\\
% 8--9 &   0.5105 &   0.5339 &   0.5628\\
% 9--10 &    0.3728 &   0.3897 &   0.2421\\
% 10--11 &   0.4400 &   0.3487 &   0.4488\\
% 11--12 &   0.4279 &   0.4098 &   0.4582\\
% 12--13 &   0.2374 &   0.2106 &   0.2939\\
% 13--14 &   0.1625 &   0.2037 &   0.1122\\
% 14--15 &   0.2809 &   0.1788 &   0.2536\\
% 15--16 &   0.1053 &   0.1839 &   0.0649\\
% 16--17 &   0.3062 &   0.2259 &   0.4022\\
% \end{tabular}
% \caption{Service coverage for all three modes in percent.}
% \end{table}

\subsubsection{Speed}

\printfile{CORDES/Mode1/results/DLSCH_throughput_speed.pdf}
\printfile{CORDES/Mode2/results/DLSCH_throughput_speed.pdf}

\printfile{CORDES/Mode6/results/DLSCH_throughput_speed.pdf}
\printfile{CORDES/Mode6/results/UL_throughput_speed.pdf}

% \begin{table}
% \centering
% \begin{tabular}{l|l|l|l}
% speed (m/s) & Mode 1 & Mode 2 & Mode 6\\
% \hline
% 0--5   &   0.3876  &  0.5360 &   0.8385\\
% 5--10  &   0.6776  &  0.4844 &   0.6915\\
% 10--15 &   0.5553  &  0.5015 &   0.6192\\
% 15--20 &   0.5694  &  0.5402 &   0.6677\\
% 20--25 &   0.5133  &  0.5376 &   0.5087\\
% 25--30 &   0.5364  &  0.5480 &   0.6111\\
% 30--35 &   0.0909  &     NaN &      NaN\\
% 35--40 &      NaN  &     NaN &      NaN\\
% \end{tabular}
% \caption{Service coverage for all three modes in percent.}
% \end{table}

\subsubsection{Ricean factor}

\printfile[Mode 2]{CORDES/results/throughput_vs_Kfactor.pdf}

\subsubsection{Extrapolation to loaded cell}

\printfile{CORDES/Mode1/results/pfair_throughput_4users_.pdf}
\printfile{CORDES/Mode1/results/pfair_throughput_cdf_4users_.pdf}

\printfile{CORDES/Mode1/results/pfair_throughput_dist_4users_.pdf}
\printfile{CORDES/Mode1/results/pfair_troughput_gps_4users_}

\printfile{CORDES/Mode2/results/pfair_throughput_4users_.pdf}
\printfile{CORDES/Mode2/results/pfair_throughput_cdf_4users_.pdf}

\printfile{CORDES/Mode2/results/pfair_throughput_dist_4users_.pdf}
\printfile{CORDES/Mode2/results/pfair_troughput_gps_4users_}

\printfile{CORDES/Mode6/results/pfair_throughput_4users_.pdf}
\printfile{CORDES/Mode6/results/pfair_throughput_cdf_4users_.pdf}

\printfile{CORDES/Mode6/results/pfair_throughput_dist_4users_.pdf}
\printfile{CORDES/Mode6/results/pfair_troughput_gps_4users_.jpg}

\printfile{CORDES/Mode2/results/pfair_throughput_4users_mode1_2Rx.pdf}
\printfile{CORDES/Mode2/results/pfair_throughput_cdf_4users_mode1_2Rx.pdf}

\printfile{CORDES/Mode2/results/pfair_throughput_dist_4users_mode1_2Rx.pdf}
\printfile{CORDES/Mode2/results/pfair_troughput_gps_4users_mode1_2Rx}

\printfile{CORDES/Mode2/results/pfair_throughput_4users_mode2_2Rx.pdf}
\printfile{CORDES/Mode2/results/pfair_throughput_cdf_4users_mode2_2Rx.pdf}

\printfile{CORDES/Mode2/results/pfair_throughput_dist_4users_mode2_2Rx.pdf}
\printfile{CORDES/Mode2/results/pfair_troughput_gps_4users_mode2_2Rx}

\printfile{CORDES/Mode2/results/pfair_throughput_4users_mode6_feedbackq_2Rx.pdf}
\printfile{CORDES/Mode2/results/pfair_throughput_cdf_4users_mode6_feedbackq_2Rx.pdf}

\printfile{CORDES/Mode2/results/pfair_throughput_dist_4users_mode6_feedbackq_2Rx.pdf}
\printfile{CORDES/Mode2/results/pfair_troughput_gps_4users_mode6_feedbackq_2Rx}


\subsubsection{Service Coverage}

\printfile{CORDES/Mode2/results/service_coverage_dl.pdf}
\printfile{CORDES/Mode2/results/service_coverage_ul.pdf}

\printfile{CORDES/Mode2/results/service_coverage_dl_rx_rssi.pdf}

\subsubsection{Distance travelled}
\label{sec:dist_travelled_cordes}


\subsubsection*{Subset 1}

\printfile[Map of zoom 1]{CORDES/results/all_modes_comparison_rssi_dBm.jpg}
\printfile[RSSI comparison of zoom 1]{CORDES/results/all_modes_comparison_rssi_dBm.pdf}

\printfile[DLSCH throughput of zoom 1]{CORDES/results/all_modes_comparison_troughput_distance_travelled.pdf}
\printfile[Ideal DLSCH throughput of zoom 1]{CORDES/results/all_modes_comparison_throughput_ideal_distance_travelled.pdf}

\printfile[DLSCH CQI of zoom 1]{CORDES/results/all_modes_comparison_cqi_distance_travelled.pdf}

\subsubsection*{Subset 2}

\printfile[Map of zoom 2]{CORDES/results/all_modes_comparison2_rssi_dBm.jpg}
\printfile[RSSI comparison of zoom 2]{CORDES/results/all_modes_comparison2_rssi_dBm.pdf}

\printfile[DLSCH throughput of zoom 2]{CORDES/results/all_modes_comparison2_troughput_distance_travelled.pdf}
\printfile[Ideal DLSCH throughput of zoom 2]{CORDES/results/all_modes_comparison2_throughput_ideal_distance_travelled.pdf}

\printfile[DLSCH CQI of zoom 2]{CORDES/results/all_modes_comparison2_cqi_distance_travelled.pdf}

\subsubsection*{Subset 3}

\printfile[Map of zoom 3]{CORDES/results/all_modes_comparison3_rssi_dBm.jpg}
\printfile[RSSI comparison of zoom 3]{CORDES/results/all_modes_comparison3_rssi_dBm.pdf}

\printfile[DLSCH throughput of zoom 3]{CORDES/results/all_modes_comparison3_troughput_distance_travelled.pdf}
\printfile[PBCH FER of zoom 3]{CORDES/results/all_modes_comparison3_pbch_fer_distance_travelled.pdf}

\subsection{L0 (uncoded) throughput}

\printfile{CORDES/results/DLSCH_uncoded_throughput_cdf_comparison.pdf}
\printfile{CORDES/results/DLSCH_uncoded_throughput_connected_cdf_comparison.pdf}

\printfile{CORDES/Mode6/results/UL_uncoded_throughput_cdf_comparison.pdf}

\section{Penne}
\label{sec:penne}

\subsection{Interference measurements}

\printfile[Interference at eNb]{PENNE/Interference/results/UL_I0_dBm.pdf}
\printfile[Interference at UE]{PENNE/Interference/results/RX_RSSI_dBm_gps.jpg}


\subsection{RX RSSI}


% \subsubsection{Coverage}
% \printfile{PENNE/Coverage/results/RX_RSSI_dBm_gps.jpg}
% \printfile{PENNE/Coverage/results/RX_RSSI_dBm.pdf}
% 
% \printfile{PENNE/Coverage/results/UL_RSSI_dBm_gps.jpg}
% \printfile{PENNE/Coverage/results/UL_RSSI_dBm.pdf}

\subsubsection{Mode 1}
\printfile{PENNE/Mode1/results/RX_RSSI_dBm_gps.jpg}
\printfile{PENNE/Mode1/results/RX_RSSI_dBm.pdf}

\printfile{PENNE/Mode1/results/UL_RSSI_dBm_gps.jpg}
\printfile{PENNE/Mode1/results/UL_RSSI_dBm.pdf}

\subsubsection{Mode2}

\printfile{PENNE/Mode2/results/RX_RSSI_dBm_gps.jpg}
\printfile{PENNE/Mode2/results/RX_RSSI_dBm.pdf}

\printfile{PENNE/Mode2/results/UL_RSSI_dBm_gps.jpg}
\printfile{PENNE/Mode2/results/UL_RSSI_dBm.pdf}

\subsubsection{Mode6}
\printfile{PENNE/Mode6/results/RX_RSSI_dBm_gps.jpg}
\printfile{PENNE/Mode6/results/RX_RSSI_dBm.pdf}

\printfile{PENNE/Mode6/results/UL_RSSI_dBm_gps.jpg}
\printfile{PENNE/Mode6/results/UL_RSSI_dBm.pdf}

\subsubsection{Path loss}

\printfile{PENNE/results/RX_RSSI_dBm_dist_bars.pdf}
\printfile{PENNE/results/RX_RSSI_dBm_dist_with_PL.pdf}


\subsection{PBCH comparison}
We show the frame error rate on the PBCH in mode 1 and 2. A more detailed comparison of some interresing parts of the routes will be given in subsection \ref{sec:dist_travelled_penne}.

\printfile{PENNE/Mode1/results/PBCH_FER.pdf}
\printfile{PENNE/Mode1/results/PBCH_fer_gps.jpg}

\printfile{PENNE/Mode2/results/PBCH_FER.pdf}
\printfile{PENNE/Mode2/results/PBCH_fer_gps.jpg}

\subsection{Rice Factor}

\printfile{PENNE/Mode2/results/K_factor_gps.jpg}

\subsection{UE mode comparison}

\printfile{PENNE/Mode1/results/UE_mode_gps.jpg}
\printfile{PENNE/Mode2/results/UE_mode_gps.jpg}

\printfile{PENNE/Mode2_OFDMA/results/UE_mode_gps.jpg}
\printfile{PENNE/Mode6/results/UE_mode_gps.jpg}

\subsection{L1 (coded) Throughput}


\subsubsection{CDF}

% \printfile{Mode1/results/DLSCH_throughput_cdf_comparison.pdf}
% \printfile{Mode2/results/DLSCH_throughput_cdf_comparison.pdf}
% \printfile{Mode6/results/DLSCH_throughput_cdf_comparison.pdf}

\printfile{PENNE/results/throughput_cdf_comparison.pdf}
\printfile{PENNE/results/throughput_connected_cdf_comparison.pdf}

\printfile{PENNE/results/throughput_cdf_comparison_fdd.pdf}

\printfile{PENNE/results/UL_throughput_cdf_comparison.pdf}
\printfile{PENNE/results/UL_throughput_cdf_comparison_fdd.pdf}

\subsubsection{Time}

\printfile{PENNE/Mode1/results/DLSCH_throughput.pdf}
\printfile{PENNE/Mode2/results/DLSCH_throughput.pdf}

\printfile{PENNE/Mode6/results/DLSCH_throughput.pdf}
\printfile{PENNE/Mode2/results/coded_throughput_time_1stRx.pdf}

\printfile{PENNE/Mode2/results/coded_throughput_time_2ndRx.pdf}
\printfile{PENNE/Mode2/results/coded_throughput_time_2Rx.pdf}

\printfile{PENNE/Mode2/results/UL_throughput.pdf}


\subsubsection{Map}

\printfile{PENNE/Mode1/results/DLSCH_troughput_gps.jpg}
\printfile{PENNE/Mode2/results/DLSCH_troughput_gps.jpg}

\printfile{PENNE/Mode6/results/DLSCH_troughput_gps.jpg}
\printfile{PENNE/Mode2/results/UL_throughput_gps.jpg}

\printfile{PENNE/Mode2/results/coded_throughput_Alamouti_gps_2Rx.jpg}
\printfile{PENNE/Mode2/results/coded_throughput_SISO_gps_2Rx.jpg}

\printfile{PENNE/Mode2/results/coded_throughput_feedbackBeamforming_gps_2Rx.jpg}
\printfile{PENNE/Mode2/results/coded_throughput_optBeamforming_gps_2Rx.jpg}

\printfile{PENNE/Mode2/results/UL_throughput_ideal_1Rx_gps.jpg}
\printfile{PENNE/Mode2/results/UL_throughput_ideal_2Rx_gps.jpg}

\subsubsection{Distance}

\printfile{PENNE/Mode1/results/DLSCH_throughput_dist.pdf}
\printfile{PENNE/Mode2/results/DLSCH_throughput_dist.pdf}

\printfile{PENNE/Mode6/results/DLSCH_throughput_dist.pdf}
\printfile{PENNE/Mode2/results/UL_throughput_dist.pdf}

%\printfile{Mode2/results/ideal_throughput_dist_mode1_1stRx.pdf}
%\printfile{Mode2/results/ideal_throughput_dist_mode1_2ndRx.pdf}

\printfile{PENNE/Mode2/results/ideal_throughput_dist_mode1_2Rx.pdf}
%\printfile{Mode2/results/ideal_throughput_dist_mode2_1stRx.pdf}
%
%\printfile{Mode2/results/ideal_throughput_dist_mode2_2ndRx.pdf}
\printfile{PENNE/Mode2/results/ideal_throughput_dist_mode2_2Rx.pdf}

%\printfile{Mode2/results/ideal_throughput_dist_mode6_feedbackq_1stRx.pdf}
\printfile{PENNE/Mode2/results/ideal_throughput_dist_mode6_feedbackq_2Rx.pdf}
%
%\printfile{Mode2/results/ideal_throughput_dist_mode6_maxq_1stRx.pdf}
\printfile{PENNE/Mode2/results/ideal_throughput_dist_mode6_maxq_2Rx.pdf}

\printfile{PENNE/Mode2/results/UL_throughput_ideal_1Rx_dist.pdf}
\printfile{PENNE/Mode2/results/UL_throughput_ideal_2Rx_dist.pdf}

\subsubsection{Speed}

\printfile{PENNE/Mode1/results/DLSCH_throughput_speed.pdf}
\printfile{PENNE/Mode2/results/DLSCH_throughput_speed.pdf}

\printfile{PENNE/Mode6/results/DLSCH_throughput_speed.pdf}
\printfile{PENNE/Mode2/results/UL_throughput_speed.pdf}

\subsubsection{Extrapolation to loaded cell}

\printfile{PENNE/Mode1/results/pfair_throughput_4users_.pdf}
\printfile{PENNE/Mode1/results/pfair_throughput_cdf_4users_.pdf}

\printfile{PENNE/Mode1/results/pfair_throughput_dist_4users_.pdf}
\printfile{PENNE/Mode1/results/pfair_troughput_gps_4users_}

\printfile{PENNE/Mode2/results/pfair_throughput_4users_.pdf}
\printfile{PENNE/Mode2/results/pfair_throughput_cdf_4users_.pdf}

\printfile{PENNE/Mode2/results/pfair_throughput_dist_4users_.pdf}
\printfile{PENNE/Mode2/results/pfair_troughput_gps_4users_}

\printfile{PENNE/Mode6/results/pfair_throughput_4users_.pdf}
\printfile{PENNE/Mode6/results/pfair_throughput_cdf_4users_.pdf}

\printfile{PENNE/Mode6/results/pfair_throughput_dist_4users_.pdf}
\printfile{PENNE/Mode6/results/pfair_troughput_gps_4users_.jpg}

\printfile{PENNE/Mode2/results/pfair_throughput_4users_mode1_2Rx.pdf}
\printfile{PENNE/Mode2/results/pfair_throughput_cdf_4users_mode1_2Rx.pdf}

\printfile{PENNE/Mode2/results/pfair_throughput_dist_4users_mode1_2Rx.pdf}
\printfile{PENNE/Mode2/results/pfair_troughput_gps_4users_mode1_2Rx}

\printfile{PENNE/Mode2/results/pfair_throughput_4users_mode2_2Rx.pdf}
\printfile{PENNE/Mode2/results/pfair_throughput_cdf_4users_mode2_2Rx.pdf}

\printfile{PENNE/Mode2/results/pfair_throughput_dist_4users_mode2_2Rx.pdf}
\printfile{PENNE/Mode2/results/pfair_troughput_gps_4users_mode2_2Rx}

\printfile{PENNE/Mode2/results/pfair_throughput_4users_mode6_feedbackq_2Rx.pdf}
\printfile{PENNE/Mode2/results/pfair_throughput_cdf_4users_mode6_feedbackq_2Rx.pdf}

\printfile{PENNE/Mode2/results/pfair_throughput_dist_4users_mode6_feedbackq_2Rx.pdf}
\printfile{PENNE/Mode2/results/pfair_troughput_gps_4users_mode6_feedbackq_2Rx}


\subsubsection{Service Coverage}

\printfile{PENNE/Mode2/results/service_coverage_dl.pdf}
\printfile{PENNE/Mode2/results/service_coverage_ul.pdf}

\printfile{PENNE/Mode2/results/service_coverage_dl_rx_rssi.pdf}

\subsubsection{Distance travelled}
\label{sec:dist_travelled_penne}

\subsubsection*{Subset 1}

\printfile[Map of zoom 1]{PENNE/results/all_modes_comparison1_rssi_dBm.jpg}
\printfile[RSSI comparison of zoom 1]{PENNE/results/all_modes_comparison1_rssi_dBm.pdf}

\printfile[DLSCH throughput of zoom 1]{PENNE/results/all_modes_comparison1_troughput_distance_travelled.pdf}
\printfile[Ideal DLSCH throughput of zoom 1]{PENNE/results/all_modes_comparison1_throughput_ideal_distance_travelled.pdf}

\printfile[DLSCH CQI of zoom 1]{PENNE/results/all_modes_comparison1_cqi_distance_travelled.pdf}
\printfile[PBCH FER of zoom 1]{PENNE/results/all_modes_comparison1_pbch_fer_distance_travelled.pdf}

\subsubsection*{Subset 2}

\printfile[Map of zoom 2]{PENNE/results/all_modes_comparison2_rssi_dBm.jpg}
\printfile[RSSI comparison of zoom 2]{PENNE/results/all_modes_comparison2_rssi_dBm.pdf}

\printfile[DLSCH throughput of zoom 2]{PENNE/results/all_modes_comparison2_troughput_distance_travelled.pdf}
\printfile[Ideal DLSCH throughput of zoom 1]{PENNE/results/all_modes_comparison2_throughput_ideal_distance_travelled.pdf}

\printfile[DLSCH CQI of zoom 2]{PENNE/results/all_modes_comparison2_cqi_distance_travelled.pdf}
\printfile[PBCH FER of zoom 2]{PENNE/results/all_modes_comparison2_pbch_fer_distance_travelled.pdf}

\subsubsection*{Subset 3}

\printfile[Map of zoom 3]{PENNE/results/all_modes_comparison3_rssi_dBm.jpg}
\printfile[RSSI comparison of zoom 3]{PENNE/results/all_modes_comparison3_rssi_dBm.pdf}

\printfile[DLSCH throughput of zoom 3]{PENNE/results/all_modes_comparison3_troughput_distance_travelled.pdf}
\printfile[Ideal DLSCH throughput of zoom 3]{PENNE/results/all_modes_comparison3_throughput_ideal_distance_travelled.pdf}

\printfile[DLSCH CQI of zoom 3]{PENNE/results/all_modes_comparison3_cqi_distance_travelled.pdf}
\printfile[PBCH FER of zoom 3]{PENNE/results/all_modes_comparison3_pbch_fer_distance_travelled.pdf}


\section{Ambialet}
\label{sec:ambialet}

\subsection{Interference measurements}

\printfile[Interference at eNb]{AMBIALET/Interference/results/UL_I0_dBm.pdf}
\printfile[Interference at UE]{AMBIALET/Interference/results/RX_RSSI_dBm_gps.jpg}


\subsection{RX RSSI}


% \subsubsection{Coverage}
% \printfile{AMBIALET/Coverage/results/RX_RSSI_dBm_gps.jpg}
% \printfile{AMBIALET/Coverage/results/RX_RSSI_dBm.pdf}
% 
% \printfile{AMBIALET/Coverage/results/UL_RSSI_dBm_gps.jpg}
% \printfile{AMBIALET/Coverage/results/UL_RSSI_dBm.pdf}

\subsubsection{Mode 1}
\printfile{AMBIALET/Mode1/results/RX_RSSI_dBm_gps.jpg}
\printfile{AMBIALET/Mode1/results/RX_RSSI_dBm.pdf}

\printfile{AMBIALET/Mode1/results/UL_RSSI_dBm_gps.jpg}
\printfile{AMBIALET/Mode1/results/UL_RSSI_dBm.pdf}

\subsubsection{Mode2}

\printfile{AMBIALET/Mode2/results/RX_RSSI_dBm_gps.jpg}
\printfile{AMBIALET/Mode2/results/RX_RSSI_dBm.pdf}

\printfile{AMBIALET/Mode2/results/UL_RSSI_dBm_gps.jpg}
\printfile{AMBIALET/Mode2/results/UL_RSSI_dBm.pdf}

\subsubsection{Mode6}
\printfile{AMBIALET/Mode6/results/RX_RSSI_dBm_gps.jpg}
\printfile{AMBIALET/Mode6/results/RX_RSSI_dBm.pdf}

\printfile{AMBIALET/Mode6/results/UL_RSSI_dBm_gps.jpg}
\printfile{AMBIALET/Mode6/results/UL_RSSI_dBm.pdf}

\subsubsection{Path loss}

\printfile{AMBIALET/results/RX_RSSI_dBm_dist_bars.pdf}
\printfile{AMBIALET/results/RX_RSSI_dBm_dist_with_PL.pdf}


\subsection{PBCH comparison}
We show the frame error rate on the PBCH in mode 1 and 2. A more detailed comparison of some interresing parts of the routes will be given in subsection \ref{sec:dist_travelled_ambialet}.

\printfile{AMBIALET/Mode1/results/PBCH_FER.pdf}
\printfile{AMBIALET/Mode1/results/PBCH_fer_gps.jpg}

\printfile{AMBIALET/Mode2/results/PBCH_FER.pdf}
\printfile{AMBIALET/Mode2/results/PBCH_fer_gps.jpg}

\subsection{Rice Factor}

\printfile{AMBIALET/Mode2/results/K_factor_gps.jpg}

\subsection{UE mode comparison}

\printfile{AMBIALET/Mode1/results/UE_mode_gps.jpg}
\printfile{AMBIALET/Mode2/results/UE_mode_gps.jpg}

\printfile{AMBIALET/Mode6/results/UE_mode_gps.jpg}

\subsection{L1 (coded) Throughput}


\subsubsection{CDF}

% \printfile{Mode1/results/DLSCH_throughput_cdf_comparison.pdf}
% \printfile{Mode2/results/DLSCH_throughput_cdf_comparison.pdf}
% \printfile{Mode6/results/DLSCH_throughput_cdf_comparison.pdf}

\printfile{AMBIALET/results/throughput_cdf_comparison.pdf}
\printfile{AMBIALET/results/throughput_connected_cdf_comparison.pdf}

\printfile{AMBIALET/results/throughput_cdf_comparison_fdd.pdf}

\printfile{AMBIALET/results/UL_throughput_cdf_comparison.pdf}
\printfile{AMBIALET/results/UL_throughput_cdf_comparison_fdd.pdf}

\subsubsection{Time}

\printfile{AMBIALET/Mode1/results/DLSCH_throughput.pdf}
\printfile{AMBIALET/Mode2/results/DLSCH_throughput.pdf}

\printfile{AMBIALET/Mode6/results/DLSCH_throughput.pdf}
\printfile{AMBIALET/Mode2/results/coded_throughput_time_1stRx.pdf}

\printfile{AMBIALET/Mode2/results/coded_throughput_time_2ndRx.pdf}
\printfile{AMBIALET/Mode2/results/coded_throughput_time_2Rx.pdf}

\printfile{AMBIALET/Mode2/results/UL_throughput.pdf}


\subsubsection{Map}

\printfile{AMBIALET/Mode1/results/DLSCH_troughput_gps.jpg}
\printfile{AMBIALET/Mode2/results/DLSCH_troughput_gps.jpg}

\printfile{AMBIALET/Mode6/results/DLSCH_troughput_gps.jpg}
\printfile{AMBIALET/Mode2/results/UL_throughput_gps.jpg}

\printfile{AMBIALET/Mode2/results/coded_throughput_Alamouti_gps_2Rx.jpg}
\printfile{AMBIALET/Mode2/results/coded_throughput_SISO_gps_2Rx.jpg}

\printfile{AMBIALET/Mode2/results/coded_throughput_feedbackBeamforming_gps_2Rx.jpg}
\printfile{AMBIALET/Mode2/results/coded_throughput_optBeamforming_gps_2Rx.jpg}

\printfile{AMBIALET/Mode2/results/UL_throughput_ideal_1Rx_gps.jpg}
\printfile{AMBIALET/Mode2/results/UL_throughput_ideal_2Rx_gps.jpg}

\subsubsection{Distance}

\printfile{AMBIALET/Mode1/results/DLSCH_throughput_dist.pdf}
\printfile{AMBIALET/Mode2/results/DLSCH_throughput_dist.pdf}

\printfile{AMBIALET/Mode6/results/DLSCH_throughput_dist.pdf}
\printfile{AMBIALET/Mode2/results/UL_throughput_dist.pdf}

\printfile{AMBIALET/Mode2/results/ideal_throughput_dist_mode1_2Rx.pdf}
\printfile{AMBIALET/Mode2/results/ideal_throughput_dist_mode2_2Rx.pdf}

\printfile{AMBIALET/Mode2/results/ideal_throughput_dist_mode6_feedbackq_2Rx.pdf}
\printfile{AMBIALET/Mode2/results/ideal_throughput_dist_mode6_maxq_2Rx.pdf}

\printfile{AMBIALET/Mode2/results/UL_throughput_ideal_1Rx_dist.pdf}
\printfile{AMBIALET/Mode2/results/UL_throughput_ideal_2Rx_dist.pdf}

\subsubsection{Speed}

\printfile{AMBIALET/Mode1/results/DLSCH_throughput_speed.pdf}
\printfile{AMBIALET/Mode2/results/DLSCH_throughput_speed.pdf}

\printfile{AMBIALET/Mode6/results/DLSCH_throughput_speed.pdf}
\printfile{AMBIALET/Mode2/results/UL_throughput_speed.pdf}

\subsubsection{Extrapolation to loaded cell}

\printfile{AMBIALET/Mode1/results/pfair_throughput_4users_.pdf}
\printfile{AMBIALET/Mode1/results/pfair_throughput_cdf_4users_.pdf}

\printfile{AMBIALET/Mode1/results/pfair_throughput_dist_4users_.pdf}
\printfile{AMBIALET/Mode1/results/pfair_troughput_gps_4users_}

\printfile{AMBIALET/Mode2/results/pfair_throughput_4users_.pdf}
\printfile{AMBIALET/Mode2/results/pfair_throughput_cdf_4users_.pdf}

\printfile{AMBIALET/Mode2/results/pfair_throughput_dist_4users_.pdf}
\printfile{AMBIALET/Mode2/results/pfair_troughput_gps_4users_}

\printfile{AMBIALET/Mode6/results/pfair_throughput_4users_.pdf}
\printfile{AMBIALET/Mode6/results/pfair_throughput_cdf_4users_.pdf}

\printfile{AMBIALET/Mode6/results/pfair_throughput_dist_4users_.pdf}
\printfile{AMBIALET/Mode6/results/pfair_troughput_gps_4users_.jpg}

\printfile{AMBIALET/Mode2/results/pfair_throughput_4users_mode1_2Rx.pdf}
\printfile{AMBIALET/Mode2/results/pfair_throughput_cdf_4users_mode1_2Rx.pdf}

\printfile{AMBIALET/Mode2/results/pfair_throughput_dist_4users_mode1_2Rx.pdf}
\printfile{AMBIALET/Mode2/results/pfair_troughput_gps_4users_mode1_2Rx}

\printfile{AMBIALET/Mode2/results/pfair_throughput_4users_mode2_2Rx.pdf}
\printfile{AMBIALET/Mode2/results/pfair_throughput_cdf_4users_mode2_2Rx.pdf}

\printfile{AMBIALET/Mode2/results/pfair_throughput_dist_4users_mode2_2Rx.pdf}
\printfile{AMBIALET/Mode2/results/pfair_troughput_gps_4users_mode2_2Rx}

\printfile{AMBIALET/Mode2/results/pfair_throughput_4users_mode6_feedbackq_2Rx.pdf}
\printfile{AMBIALET/Mode2/results/pfair_throughput_cdf_4users_mode6_feedbackq_2Rx.pdf}

\printfile{AMBIALET/Mode2/results/pfair_throughput_dist_4users_mode6_feedbackq_2Rx.pdf}
\printfile{AMBIALET/Mode2/results/pfair_troughput_gps_4users_mode6_feedbackq_2Rx}

\subsubsection{Service Coverage}

\printfile{AMBIALET/Mode2/results/service_coverage_dl.pdf}
\printfile{AMBIALET/Mode2/results/service_coverage_ul.pdf}

\printfile{AMBIALET/Mode2/results/service_coverage_dl_rx_rssi.pdf}

\subsubsection{Distance travelled}
\label{sec:dist_travelled_ambialet}

\subsubsection*{Subset 1}

\printfile[Map of zoom 1]{AMBIALET/results/all_modes_comparison1_rssi_dBm.jpg}
\printfile[RSSI comparison of zoom 1]{AMBIALET/results/all_modes_comparison1_rssi_dBm.pdf}

\printfile[DLSCH throughput of zoom 1]{AMBIALET/results/all_modes_comparison1_troughput_distance_travelled.pdf}
\printfile[Ideal DLSCH throughput of zoom 1]{AMBIALET/results/all_modes_comparison1_throughput_ideal_distance_travelled.pdf}

\printfile[DLSCH CQI of zoom 1]{AMBIALET/results/all_modes_comparison1_cqi_distance_travelled.pdf}
\printfile[PBCH FER of zoom 1]{AMBIALET/results/all_modes_comparison1_pbch_fer_distance_travelled.pdf}

\subsubsection*{Subset 2}

\printfile[Map of zoom 2]{AMBIALET/results/all_modes_comparison2_rssi_dBm.jpg}
\printfile[RSSI comparison of zoom 2]{AMBIALET/results/all_modes_comparison2_rssi_dBm.pdf}

\printfile[DLSCH throughput of zoom 2]{AMBIALET/results/all_modes_comparison2_troughput_distance_travelled.pdf}
\printfile[Ideal DLSCH throughput of zoom 1]{AMBIALET/results/all_modes_comparison2_throughput_ideal_distance_travelled.pdf}

\printfile[DLSCH CQI of zoom 2]{AMBIALET/results/all_modes_comparison2_cqi_distance_travelled.pdf}
\printfile[PBCH FER of zoom 2]{AMBIALET/results/all_modes_comparison2_pbch_fer_distance_travelled.pdf}

\subsubsection*{Subset 3}

\printfile[Map of zoom 3]{AMBIALET/results/all_modes_comparison3_rssi_dBm.jpg}
\printfile[RSSI comparison of zoom 3]{AMBIALET/results/all_modes_comparison3_rssi_dBm.pdf}

\printfile[DLSCH throughput of zoom 3]{AMBIALET/results/all_modes_comparison3_troughput_distance_travelled.pdf}
\printfile[Ideal DLSCH throughput of zoom 3]{AMBIALET/results/all_modes_comparison3_throughput_ideal_distance_travelled.pdf}

\printfile[DLSCH CQI of zoom 3]{AMBIALET/results/all_modes_comparison3_cqi_distance_travelled.pdf}
\printfile[PBCH FER of zoom 3]{AMBIALET/results/all_modes_comparison3_pbch_fer_distance_travelled.pdf}


\section{Conclusions}


\bibliographystyle{IEEEtranS} 
\bibliography{print_files}


\end{document}          
